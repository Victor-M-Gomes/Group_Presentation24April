\message{ !name(presentation.tex)}\documentclass[utf8]{beamer}	
\usetheme{lfcr}           % Use metropolis theme

\usepackage[brazil,english,francais]{babel}

\usepackage{natbib}
\setcitestyle{round}

\usepackage{lmodern}
\usepackage{datetime} 

\usepackage{ulem}

\usepackage{tikz}
\usetikzlibrary{shapes.geometric, arrows, shapes.misc, shadows.blur}
\usepackage[most]{tcolorbox}

\usepackage{xcolor}

%%%% Defining colors
\definecolor{RoyalBlue}{rgb}{0.0, 0.14, 0.4}
\definecolor{OliveGreen}{rgb}{0.33, 0.42, 0.18}
\definecolor{Burgundy}{rgb}{0.5, 0.0, 0.13}
\definecolor{Black}{rgb}{0.0, 0.0, 0.0}
\definecolor{Blue}{rgb}{0.0, 0.53, 0.74}

\usepackage{natbib}

\usepackage{graphicx} % Allows including images
\usepackage{booktabs} % Allows the use of \toprule

\usepackage{subfig}
\usepackage{stackengine}

\usepackage{float}
\usepackage{placeins}   %Binds figures to its respective sections

\usepackage{cases}

\usepackage{varwidth}

\usepackage{multirow}

\usepackage{textcomp} %Text Com­n­in fonts, which pro­vie many text 
		      %sym­ols (such as baht, bul­lt, copy­riht, mu­si­cl­icalnote, 
		      %onequar­er, sec­ton, and yen), in the TS1 en­cocoding.

\usepackage{algpseudocode}

\usepackage{hyperref}

\usepackage{pgfgantt} % Gantt charts - for planning

\setbeamercovered{transparent} % For transparency effects with overlay commands
%
%
%
%% figure files' paths
\graphicspath{%
{../KitPartenairesUPPA/}% logo UPPA
{../logos/}% misc logos
{Figures/}
{Figures/VEr/}
{Figures/VEz/}
}

\title{Automation of electric acquisition for the new experimental setup}
\date{\today}
\author{Victor MARTINS GOMES \\[3mm] Daniel BRITO (PhD supervisor) \\ \and H\'{e}l\`{e}ne BARUCQ (PhD co-supervisor)}
\begin{document}

\message{ !name(presentation.tex) !offset(-3) }

  \begin{frame}
	\titlepage
  \end{frame}
 
%
\begin{frame}{Automation - general view}
      \textbf{Schema of what needs to be done}
      \begin{center}
            \includegraphics[width=0.85\textwidth]{schema.pdf}
      \end{center}
\end{frame}
%
\begin{frame}{Automation - general view}
      \textbf{Main points}
      \begin{itemize}
            \item {Control oscilloscope (config. and recording mainly) through 
Python routines}
            \item {Send signal to RaspberryPi to change relays}
            \item {Operate the electronic card through SPI using the RaspberryPi' GPIO pins}
            \item {Send signal to oscilloscope to start recording}
      \end{itemize}
\end{frame}
%%%%
\begin{frame}{Automation Planning}
	\hspace*{-0.5cm}
	\resizebox{11.5cm}{!}{
	\begin{ganttchart}[
	x unit=0.45cm,
	y unit title=0.7cm,
	y unit chart=0.9cm,
	vgrid={draw=none, dotted},
	time slot format=isodate,
	time slot unit=day,
	calendar week text = {W\currentweek{}},
	title/.append style={draw=none, fill=RoyalBlue!50!black},
	title label font= \sffamily\bfseries\Large\color{white},
	title label node/.append style={below=-1.6ex},
	title left shift=.05,
	title right shift=-.05,
	title height=1,
	bar/.append style={draw=none, fill=OliveGreen!75},
	bar height=.6,
	bar label font=\Large\color{black!80},
	group right shift=0,
	group top shift=.6,
	group height=.3,
	group peaks height=.2,
	group label font=\sffamily\bfseries\LARGE\color{black},
	bar incomplete/.append style={fill=Burgundy},
	bar progress label font=\Large\color{Black},
	group progress label font=\Large\color{Black},
	milestone progress label font=\Large\color{Black},
	milestone label font=\Large\color{Blue},
	bar progress label font=\Large\color{Black},
	progress=today,
	today=2020-02-20
	]{2020-01-30}{2020-03-13}
	\gantttitlecalendar{year, month=name, week} \\
	\ganttbar[
	bar progress label font=\Large\color{OliveGreen!75},
	bar progress label node/.append style={right=4pt},
	progress=40,
	bar label font=\LARGE\color{OliveGreen},
	name=pp
	]{Overall}{2020-01-30}{2020-03-13} \\
	\ganttset{link/.style={black, thick, -to}}
	\ganttbar[name=osc,
	progress=70]{Oscilloscope}{2020-01-30}{2020-03-13} \\
	\ganttbar[name=elec,
	progress=50]{Elec. card}{2020-01-30}{2020-03-13} \\
	\ganttbar[name=las,
	progress=90]{Acoustic}{2020-01-30}{2020-03-13} \\
	progress=40]{Elec. card}{2020-01-30}{2020-03-13} \\
	\ganttbar[name=tav,
	progress=30]{Tests and vali.}{2020-01-30}{2020-03-13} \\
	\end{ganttchart}
	}

	\textbf{Dates going from: 2020-01-30 until 2020-03-13}
\end{frame}
%
\begin{frame}{Automation Planning}
	\textbf{Points:}
	\begin{itemize}
		\item {Oscilloscope connection has no easy-to-use user interface.
 Possibility of bugs I have not yet seen.}
		\item {Control over GPIO pins works but SPI is not quite clear.
			\begin{itemize}
				\item {The right cable config.}
				\item {What 8-bit message do I send to change drivers/relays.}
			\end{itemize}}
		\item {Oscilloscope control could be done with LabView to make it easier.}
	\end{itemize}
\end{frame}
%
\begin{frame}{Tests to be performed}
	\begin{itemize}
		\item Check if all relays work properly - \textcolor{blue}{DONE}
		\item Test box attenuation/plastic velocity/etc \textcolor{blue}{ONGOING}
		\item Water-filled box with dipole source to check electrodes' beahviour to electromagnetic sources
		\item Sand-filled box
		\item Sensitivity to the Layer response 
	\end{itemize}
\end{frame}
%
\begin{frame}{Perspectives to new experimentl setup}
	\textbf{General:}
	\begin{enumerate}
		\item Greater SNR;
		\item Faster acquisition
		\item Greater spatial precision of electric measurements due to more rigid electrodes;
		\item Ensure repeatability;
		\item More precision when studying the converted wave.
	\end{enumerate}
\end{frame}
%%%%%%%%%%%%%%%%%%%%%%%%%%%%%%%%%%%%%%%%%%%%%%%%%%%%%%%%%%%%%%%%%%%%%%%%%
  
  %%%%%%%%%%%%%%%%%%%%%%%
  % DO NOT CHANGE BELOW %
  %%%%%%%%%%%%%%%%%%%%%%%
  \begin{frame}{}	
	\centering 
	\begin{beamercolorbox}[wd=\paperwidth, ht=1.2cm]{frametitle} 
	       \begin{tikzpicture}
		   \useasboundingbox(0,0) rectangle(\the\paperwidth,1.5);
		   {
	    	   \draw[preaction={fill=black,opacity=.4,transform canvas={xshift=0.0mm,yshift=-0.5mm}}][draw=white,fill=myblue] (-0.1,-1) -- (-0.1,1.5) -- (19,1.5) -- (19,-0.25);
		   \node[anchor=base] at (6.12875,0) {\usebeamerfont{frametitle}\huge Thank you for your attention!};
		   }
	       \end{tikzpicture}
        \end{beamercolorbox}
  \end{frame}

\message{ !name(presentation.tex) !offset(-5) }

\end{document}
